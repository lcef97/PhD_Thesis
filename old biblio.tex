\documentclass{article}
\usepackage{url}
\begin{document}

Old bibliography in text format. Missing Silverman (1986).
\begin{thebibliography}{99}
%\bibitem{Banerjee}
%Banerjee, S., Carlin, B.P.,  Gelfand, A.E.: . Hierarchical modelling and Analysis for Spatial Data(2014, 2nd ed.). Chapman and Hall/CRC. https://doi.org/10.1201/b17115

\bibitem[Agasisti and Vittadini  2012]{Agasisti}
Agasisti, T., Vittadini, G. (2012). Regional economic disparities as determinants of student's achievement in Italy. Research in Applied Economics, 4(2), 33. DOI: \url{http://dx.doi.org/10.5296/rae.v4i2.1316 }

\bibitem[Azzalini and Capitanio 2013]{SN}
Azzalini, A., Capitanio, A. (2013). The skew-normal and related families (Vol. 3). Cambridge University Press.

\bibitem[Bakka et al. 2018]{INLArev} 
Bakka, H., H. Rue, G.-A. Fuglstad, A. Riebler, D. Bolin, E. Krainski, D. Simpson, and F. Lindgren (2018). Spatial modelling with R-INLA: A review. WIREs Comput Stat 10(6), 1–24; DOI: \url{https://doi.org/10.1002/wics.1443}

\bibitem[Barrett et al. 2019]{WB}
Barrett, P., Treves, A., Shmis, T., Ambasz, D. (2019). The impact of school infrastructure on learning: A synthesis of the evidence. International Development in Focus;. © Washington, DC: World Bank. \url{http://hdl.handle.net/10986/30920}

\bibitem[Benadusi et al. 2010]{Benadusi}
Benadusi, L., Fornari, R.,  Giancola, O. (2010). Così vicine, così lontane. La questione dell’equità scolastica nelle regioni italiane. Scuola Democratica: Learning for Democracy, 1, 52-79.

\bibitem[Besag 1974]{CAR}
Besag, J. (1974). Spatial Interaction and the Statistical Analysis of Lattice Systems. Journal of the Royal Statistical Society. Series B (Methodological) 36, no. 2: 192–236. DOI:\url{ https://doi.org/10.1111/j.2517-6161.1974.tb00999.x}

\bibitem[Besag et al. 1991]{BYM}
Besag, J., York, J., Mollié, A. (1991). Bayesian image restoration, with two applications in spatial statistics. Annals of the institute of statistical mathematics, 43, 1-20. DOI: \url{https://doi.org/10.1007/BF00116466}

\bibitem[Besag and Kooperberg 1995]{ICAR}
Besag, J., Kooperberg, C. (1995) On conditional and intrinsic autoregressions. Biometrika, 82(4), 733-746. DOI:\url{https://doi.org/10.1093/biomet/82.4.733}

\bibitem[Bivand et al. 2024]{spdep}
Bivand, R., Altman, M., Anselin, L., Assunção, R., Berke, O., Bernat, A., Blanchet, G. (2024). Package ‘spdep’. \textit{Spatial dependence: Weighting schemes, statistics, R package version}, 1.3-7, url \url{https://cran.r-project.org/package=spdep} 

\bibitem[Blangiardo and Cameletti 2015]{Blangiardo}
Blangiardo, M., Cameletti, M. (2015). Spatial and spatio-temporal Bayesian models with R-INLA. John Wiley and Sons. DOI:\url{https://doi.org/10.1002/9781118950203}

\bibitem[Bratti et al. 2007]{Bratti}
Bratti, M.,  Checchi, D., Filippin, A. (2007) Territorial differences in Italian students' mathematical competencies: evidence from Pisa 2003, Giornale Degli Economisti e Annali Di Economia, 66 (Anno 120)(3), 299–333, \url{https://www.jstor.org/stable/23248253}.

\bibitem[Bucci et al. 2024]{BDI}
Bucci, M., Gazzano, L., Gennari, E., Grompone, A., Ivaldi, G., Messina, G.,  Ziglio, G. (2024) Per chi suona la campan(ell)a? La dotazione di infrastrutture scolastiche in Italia (For whom the bell tolls? The availability of school infrastructure in Italy). Politica economica, 1-50, DOI: 10.32057/0.QEF.2023.0827.

\bibitem[Cefalo et al. 2024]{SchoolDataIT}
Cefalo, L., Pollice, A., Maranzano, P. (2024). SchoolDataIT: Retrieve, Harmonise and Map Open Data regarding the Italian School System, R Package Version 0.2.2, \url{https://CRAN.R-project.org/package=SchoolDataIT}

%\bibitem{Chung}
%Chung, F.: Spectral Graph Theory. CBMS Regional Conference Series,
%vol. 92. Conference Board of the Mathematical Sciences, Washington (1997)

%\bibitem{Cressie}
%Cressie, N.: Statistics for Spatial Data (Revised Edition)(1993). USA: John Wiley and Sons

\bibitem[Cliff and Ord 1981]{Cliff_Ord}
Cliff, A. D.  Ord, J. K. (1981). Spatial Processes: Models and Applications. Pion, London. 

\bibitem[Dupont et al. 2022]{Dupont}
Dupont, E., Wood, S.N., Augustin, N.H. (2022). Spatial+:A novel approach to spatial confounding. Biometrics, 78:1279–1290. DOI:\url{https://doi.org/10.1111/biom.13656}

\bibitem[Dupont et al. 2023]{DupontArXiv}
Dupont, E., Marques, I. Kneib, T.: Demystifying Spatial Confounding (2023). arXiv \url{https://doi.org/10.48550/arXiv.2309.16861}.

\bibitem[Freni-Sterrantino et al. 2018]{Freni}
Freni-Sterrantino, A., Ventrucci, M., Rue, H. (2018). A note on intrinsic conditional autoregressive models for disconnected graphs. Spatial and spatio-temporal epidemiology, 26, 25-34. DOI: \url{https://doi.org/10.1016/j.sste.2018.04.002}

\bibitem[Geisser and Eddy 1979]{Geisser}
Geisser, S., Eddy, W. F. (1979). A predictive approach to model selection. Journal of the American Statistical Association, 74(365), 153-160. DOI: \url{https://doi.org/10.1080/01621459.1979.10481632}

\bibitem[Gelfand et al. 1992]{Gelfand}
Gelfand, A. E., Dey, D. K., Chang, H. (1992). Model determination using predictive distributions with implementation via sampling-based methods. Bayesian statistics 4, 147-168. DOI:\url{https://doi.org/10.1093/oso/9780198522669.003.0009}

\bibitem[Gelfand and Vounatsu 2003]{PCAR_Gelfand}
Gelfand A.E., Vounatsou P. (2003) Proper multivariate conditional autoregressive models for spatial data analysis. Biostatistics. 2003 Jan; 4(1):11-25. PMID: 12925327, DOI: \url{https://doi.org/10.1093/biostatistics/4.1.11} 

\bibitem[Gelman et al. 2004]{Gelman}
Gelman, A., Carlin, J.B., Stern, H.S., Rubin, D.B. (2004) Bayesian Data Analysis (2nd ed.). Chapman and Hall/CRC. ISBN: 1-58488-388-X, DOI:\url{https://doi.org/10.1201/9780429258480}

\bibitem[Gelman et al. 2014]{GelmanWAIC}
Gelman, A., Hwang, J., Vehtari, A. (2014). Understanding predictive information criteria for Bayesian models. Statistics and computing, 24, 997-1016. DOI: \url{https://doi.org/10.1007/S11222-013-9416-2}

\bibitem[Giancola and Salmieri 2020]{UniromaWP1}
Giancola, O., Salmieri, L. (2020) Family Background, School-Track and Macro-Area: the Complex Chains of Education Inequalities in Italy. \url{https://hdl.handle.net/11573/1366601 }

%\bibitem[Griffith 2010]{Griffith}
%Griffith, D. A. (2010) The Moran coefficient for non-normal data. Journal of Statistical Planning and Inference, \textit{140}(11), 2980-2990

\bibitem[Gómez-Rubio 2020]{INLAbook}
Gómez-Rubio, V. (2020). Bayesian inference with INLA. Chapman and Hall/CRC. DOI: \url{https://doi.org/10.1201/9781315175584}

\bibitem[Held et al 2010]{CPOINLA}
Held, L., Schrödle, B., Rue, H. (2010). Posterior and Cross-validatory Predictive Checks: A Comparison of MCMC and INLA. In: Kneib, T., Tutz, G. (eds) Statistical Modelling and Regression Structures. Physica-Verlag HD. DOI: \url{https://doi.org/10.1007/978-3-7908-2413-1\_6} 

\bibitem[Hodges et al. 2003]{Hodges2003}
Hodges, J. S., Carlin, B. P., Fan, Q. (2023). On the precision of the conditionally autoregressive prior in spatial models. Biometrics, \textit{59}(2), 317-322. DOI: \url{https://doi.org/10.1111/1541-0420.00038}

\bibitem[Hodges and Reich 2010]{Hodges}
Hodges, JS., Reich, BJ. (2010). Adding Spatially-Correlated Errors Can Mess Up the Fixed
Effect You Love. The American Statistician, 2010, 64(4), 325–334. DOI: \url{https://doi.org/10.1198/tast.2010.10052}

\bibitem[Khan and Calder 2022]{Khan}
Khan, K., Calder, C. A. (2022). Restricted spatial regression methods: Implications for inference. Journal of the American Statistical Association, 117(537), 482-494. DOI: \url{https://doi.org//10.1080/01621459.2020.1788949}

\bibitem[Infratel SPA 2024]{BB}
Infratel Italia. Schools Dashboard, part of the Ultra-Broadband Activation plan, available at \url{https://bandaultralarga.italia.it/scuole-voucher/progetto-scuole/}, last access December 19th 2024.

\bibitem[Invalsi 2013]{Invalsi2013}
Invalsi National Assessment System (2023), Rilevazioni nazionali sugli apprendimenti 2012/13. Il quadro di sistema, url:\url{https://www.invalsi.it/snvpn2013/rapporti/Rapporto_SNV_PN_2013_DEF_11_07_2013.pdf}.

\bibitem[Invalsi 2024]{Invalsi_IS}
Invalsi - Istituto Nazionale per la Valutazione del Sistema Educativo di Istruzione e Formazione (National Institute for the Evaluation of the Education System). Municipality Standardized Assessment Data (Invalsi Censuary Survey). Invalsi Statistical Services. Available at \url{https://serviziostatistico.invalsi.it/invalsi_ss_data/dati-comunali-di-popolazione-comune-del-plesso/}, last access December 18th 2024.

\bibitem[ISTAT 2022]{InnerAreas}
ISTAT (2022). La geografia delle aree interne nel 2020: vasti territori tra potenzialità e debolezze. \url{https://www.istat.it/it/files//2022/07/FOCUS-AREE-INTERNE-2021.pdf}

\bibitem[Italian Official Journal 2007]{InvalsiLaw} 
Italian Official Journal, Law 176 of October 25th 2007. \url{https://www.gazzettaufficiale.it/atto/serie_generale/caricaDettaglioAtto/originario?atto.dataPubblicazioneGazzetta=2007-10-26&atto.codiceRedazionale=007G0195&elenco30giorni=false}

%Ministry of Economic Development, decree of 7th 2020, Italian official journal, url: \url{https://www.gazzettaufficiale.it/eli/id/2020/10/01/20A05280/sg}

%\bibitem{UniromaWP2}Locicero, A.: School Performance Gaps in Italian Regions: Estimating the Impact of Individual, School, and Territorial Factors. INVALSI 2021/22 Data Analysis, Working paper (2024)

\bibitem[Lamouroux et al. 2024]{Lamouroux}
Lamouroux, J., Geffroy, A., Leblond, S., Meyer, C.,  Albert, I. (2024). Addressing Spatial Confounding in geostatistical regression models: An R-INLA approach. arXiv preprint arXiv:2410.01530. DOI: \url{https://doi.org/10.48550/arXiv.2410.01530}

\bibitem[Mardia 1988]{Mardia}
Mardia, K.V. (1988) Multi-dimensional multivariate Gaussian Markov random fields with application to image processing. JMA 24, 265-284. DOI: \url{https://doi.org/10.1016/0047-259X(88)90040-1}

\bibitem[Martini 2020]{Invalsi2020}
Martini, A. (2020) Il Divario Nord-Sud nei Risultati delle Prove INVALSI. Invalsi Working Paper n. 52

\bibitem[Matteucci and Mignani 2014]{Matteucci}
Matteucci, M., Mignani, S. (2014). Exploring regional differences in the reading competencies of Italian students. Evaluation review, 38(3), 251-290 DOI: \url{https://doi.org/10.1177/0193841X14540289}

%\bibitem{PCAR_Kim}
%Kim, H., Sun, D., Tsutakawa, R. K. (2001), A bivariate Bayes method for improving the estimates of mortality rates with a twofold conditional autoregressive model. Journal of the American Statistical association, 96(456), 1506-1521.

\bibitem[Italian Ministry of Education 2024]{MIUR}
Italian Ministry of Education, University and Research: Portale Unico dei Dati sulla scuola (Unique School Data Portal), available at \url{https://dati.istruzione.it/opendata/}, last accessed on December 17th 2024.

\bibitem[Nobre et al. 2021]{Nobre}
Nobre, W. S., Schmidt, A. M., Pereira, J. B. (2021). On the effects of spatial confounding in hierarchical models. International Statistical Review, 89(2), 302-322. DOI: \url{https://doi.org/10.1111/insr.12407}

\bibitem[OECD 2019]{PISA}
OECD (2009). PISA Data Analysis Manual: SPSS, Second Edition, PISA, OECD Publishing, Paris, DOI: \url{https://doi.org/10.1787/9789264056275-en}

\bibitem[Palmí - Perales et al. 2021]{INLAMSM}
Palmí-Perales, F., Gómez-Rubio, V.,  Martinez-Beneito, M. A. (2021). Bayesian Multivariate Spatial Models for Lattice Data with INLA. Journal of Statistical Software, 98(2), 1–29. DOI: \url{https://doi.org/10.18637/jss.v098.i02}

\bibitem[Palmí - Perales et al. 2023]{INLAMSM2}
Palmí-Perales, F., Gómez-Rubio, V., Bivand, R. S., Cameletti, M., Rue, H. (2023). Bayesian Inference for Multivariate Spatial Models with INLA, The R Journal

\bibitem[Pettit 1990]{CPO}
Pettit, L. I. (1990). The Conditional Predictive Ordinate for the Normal Distribution. Journal of the Royal Statistical Society. Series B (Methodological) 52, no. 1: 175–84. DOI: \url{https://doi.org/10.1111/j.2517-6161.1990.tb01780.x}

\bibitem[Sørbye and Rue 2014]{Sorbye}
Sørbye, S., Rue, H. (2014) Scaling intrinsic Gaussian Markov random field priors in spatial modelling, Spatial Statistics, Vol. 8, 39-51, ISSN 2211-6753, DOI: \url{https://doi.org/10.1016/j.spasta.2013.06.004}

\bibitem[Reich et al. 2006]{RHZ}
Reich, B. J., Hodges, J. S.,  Zadnik, V.: Effects of residual smoothing on the posterior of the fixed effects in disease-mapping models. Biometrics, 62(4):1197–1206. DOI: \url{https://doi.org/10.1111/j.1541-0420.2006.00617.x}

%\bibitem[Rue and Held 2005]{GMRFs}
%Rue, H.,  Held, L. (2005). Gaussian Markov Random Fields: Theory and Applications (1st ed.). Chapman and Hall/CRC. https://doi.org/10.1201/9780203492024

\bibitem[Rue et al. 2009]{INLA}
Rue, H., Martino, S., Chopin, N. (2009)  Approximate Bayesian inference for latent Gaussian models by using integrated nested Laplace approximations. Journal of the Royal Statistical Society Series B: (Methodological), 71, no. 2: 319-392. DOI: \url{https://doi.org/10.1111/j.1467-9868.2008.00700.x}

%\bibitem{INLA2017}
%Rue, H., Riebler, A., Sørbye, S. H., Illian, J. B., Simpson, D. P.,  Lindgren, F. K.: Bayesian computing with INLA: a review. \textit{Annual Review of Statistics and Its Application}, \textit{4}(1), 395-421 (2017). 

\bibitem[Simpson et al. 2017]{PCprior}
Simpson D, Rue H, Riebler A, Martins TG, Sørbye SH (2017) Penalising Model Component Complexity: A Principled, Practical Approach to Constructing Priors. Statistical Science 32: 1–28. DOI: \url{http://dx.doi.org/10.1214/16-STS576}.

\bibitem[Spiegelhalter et al. 2002]{DIC}
Spiegelhalter, D. J., Best, N. G., Carlin, B. P., Van Der Linde, A. (2002). Bayesian measures of model complexity and fit. Journal of the royal statistical society: Series B (statistical methodology), 64(4), 583-639. DOI: \url{https://doi.org/10.1111/1467-9868.00353}

\bibitem[Urdangarin et al. 2023]{Urdangarin23}
Urdangarin, A., Goicoa, T., Ugarte, M. D. (2023) Evaluating recent methods to overcome spatial confounding, Revista Matemática Complutense 36.2: 333-360. DOI: \url{http://dx.doi.org/10.1007/s13163-022-00449-8, (}

 \bibitem[Urdangarin et al. 2024]{Urdangarin24}
Urdangarin, A., Goicoa, T., Kneib, T., Ugarte, M. D. (2024). A simplified spatial+ approach to mitigate spatial confounding in multivariate spatial areal models. Spatial Statistics, 59 (2024), 100804. DOI: \url{https://doi.org/10.1016/j.spasta.2023.100804}

\bibitem[Van Niekerk and Rue (2021)]{SNprior}
Van Niekerk, J., Rue, H. (2021) Skewed Probit Regression - Identifiability, Contraction and Reformulation. Revstat - Statistical Journal. 19. 1-22. DOI:\url{https://doi.org/10.57805/revstat.v19i1.328}

\bibitem[Van Niekerk et al. 2023]{VBINLA}
Van Niekerk, J., Krainski, E., Rustand, D., Rue, H. (2023). A new avenue for Bayesian inference with INLA. Computational Statistics and Data Analysis, 181, 107692. DOI: \url{https://doi.org/10.1016/j.csda.2023.107692}

\bibitem[Van Niekerk and Rue 2024]{VB}
Van Niekerk, J.,  Rue, H. (2024) Low-rank variational Bayes correction to the Laplace method. The Journal of Machine Learning Research, 25(62), 1-25. \url{http://jmlr.org/papers/v25/21-1405.html}

\bibitem[Wang et al. 2018]{Wang}
Wang, X., Yue, Y. R., Faraway, J. J. (2018). Bayesian regression modeling with INLA. Chapman and Hall/CRC. DOI: \url{https://doi.org/10.1201/9781351165761}

\bibitem[Watanabe 2013]{WAIC}
Watanabe, S. (2013) A widely applicable Bayesian information criterion. The Journal of Machine Learning Research, 14(1), 867-897 (2013). \url{http://jmlr.org/papers/v14/watanabe13a.html}

\bibitem[Wickham 2011]{ggplot}
Wickham, H. (2011). ggplot2. Wiley interdisciplinary reviews: computational statistics, 3(2), 180-185. R package version 3.5.1, \url{https://cran.r-project.org/web/packages/ggplot2/index.html}



\end{thebibliography}
\end{document}